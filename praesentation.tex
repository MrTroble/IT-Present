\documentclass[hyperref={pdfpagelabels=false}]{beamer}
\usepackage{lmodern}
\title{Bisphenol-A}   
\author{Jakob Bolenbach} 
\date{\today} 
\setbeamertemplate{navigation symbols}{}
\usepackage{beamerthemeshadow}
\beamersetuncovermixins{\opaqueness<1>{25}}{\opaqueness<2->{15}}
\usepackage{natbib}                  
\usepackage{bibgerm} 

\begin{document}
\begin{frame}
\titlepage
\end{frame} 

\begin{frame}
\frametitle{Inhaltsverzeichnis}
\tableofcontents
\end{frame} 


\section{Historisches} 
\subsection{Anfangszweck und Hormonelle Wirkung}
\begin{frame}
\frametitle{Anfangszweck}
\begin{itemize}
\item Ersatzstoff für Östrogen
\end{itemize}

\end{frame}
\begin{frame}
\frametitle{Hormonelle Wirkung}
Z.b. Impotenz bei Männern
\end{frame}


\section{Data-Sheet} 
\subsection{Darstellung, Aufbau und Risiken}
\begin{frame}
\frametitle{Darstellung und Aufbau}
\begin{figure}
\includegraphics[scale=.5]{BPAStrukturformel.png}
\caption{BPA-Aufbau}
\end{figure}
\end{frame}

\begin{frame}
\frametitle{Risiken}
\begin{itemize}[<+->]
\item  Hormonelle Wirkung auf Männer und Heranwachsende
\item  Hormonelle Auswirkungen in anderen Organismen
\item  Umwelt Risiken
\end{itemize} 
\end{frame}

\section{Verwendungszweck von BPA heute} 
\subsection{Vorkommen im Alltag}
\begin{frame}
\frametitle{Vorkommen in der Nahrung}
\pause
\begin{itemize}
\item \invisible<1>{Z.b. McDonalds Burger-Verpackung}
\end{itemize}

\end{frame}

\begin{frame}
\frametitle{Vorkommen in Alltagsgegenständen}
\begin{tabular}{c c}
Kontoauszug & Blue-Ray\\
\pause 
\invisible<1>{Bis 1.01.2020 im Thermopapier & Beschichtung} 
\end{tabular} 
\end{frame}


\section{Freisetzung}
\subsection{Freisetzung von BPA in Körper und Nahrung}

\begin{frame}
\frametitle{Freisetzung allgemein}
\pause
\begin{itemize}
\item \invisible<1>{Durch Wärme, Aufheizen, Säuren und Laugen}
\end{itemize}
\end{frame}
\begin{frame}
\frametitle{Freisetzung im Humanen Körper}
\pause
\begin{itemize}
\item \invisible<1>{Z.b. Durch die Magensäure}
\end{itemize}
\end{frame}

\section{Politisches}
\subsection{Wirtschaftliche Vor- und Nachteile}

\begin{frame}
\frametitle{Pro und Kontra nebeneinander gestellt }
\begin{center}


\begin{tabular}{|c|c|}
\hline
\textbf{Vorteile} & \textbf{Nachteile} \\
\hline
Herstellung ist Billig &  Gesundheitliche Probleme\\
\hline
usw. ... & ... \\
\hline
\end{tabular}
\end{center}
\end{frame}

\subsection{Regulierung durch den Staat}
\begin{frame}
\frametitle{Wie wurde/wird BPA reguliert?}
\begin{itemize}
\item Immer strenger werdende Gesetze
\item Evtl.Einführung in die REACH
\item usw.
\end{itemize}

\end{frame}

\subsection{Interview mit Herr Dr. Hüttenhofer} 
\begin{frame}
\frametitle{Das Interview mit Herr Dr. Hüttenhofer}
\end{frame}

\section{Quellen}
\begin{frame}
\frametitle{Quellen}
\begin{enumerate} 
\item $https://www.umweltbundesamt.de/themen/chemikalien/chemikalien-reach/stoffgruppen/bisphenol-atextpart-1 - 02.01.2020$
\item $https://de.wikipedia.org/wiki/Bisphenol_ADarstellung - 02.01.2020$
\item $https://utopia.de/ratgeber/bisphenol-a-bpa-chemikalie-hormonelle-wirkung/ - 02.01.2020$
\item $https://www.kpb.co.kr/resources/download/user/chn/catalog/200/BISPHENOL-A20SDS.pdf - 02.01.2020$
\item $(Buch) Bisphenol A Removal from Water and Wastewater - 02.01.2020$
\item $https://www.umweltbundesamt.de/en/tags/bisphenol-a$
\end{enumerate}
\end{frame}
\end{document}